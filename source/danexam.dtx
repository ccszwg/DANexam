% \iffalse meta-comment
%
% Copyright (C) 2018 by 蛋疼的蛋蛋 <sd44sd44@yeah.net>
%
% This file may be distributed and/or modified under the
% conditions of the LaTeX Project Public License, either
% version 1.3 of this license or (at your option) any later
% version. The latest version of this license is in:
%
% http://www.latex-project.org/lppl.txt
%
% and version 1.3 or later is part of all distributions of
% LaTeX version 2005/12/01 or later.
%
% \fi

% \iffalse
%<package>\NeedsTeXFormat{LaTeX2e}[2005/12/01]
%<package>\ProvidesPackage{danexam}
%<package> [2018/11/21 v0.1 danexam is a simple exam package]
%
%<*driver>
\documentclass[a4paper, oneside]{ltxdoc}
\usepackage{dandanstyle}
% \usepackage{danexam}

\EnableCrossrefs
\CodelineIndex
\RecordChanges
\OnlyDescription

\begin{document}
  \DocInput{danexam.dtx}
\end{document}
%</driver>
% \fi
%
% \CheckSum{0}
%
% \CharacterTable
%  {Upper-case    \A\B\C\D\E\F\G\H\I\J\K\L\M\N\O\P\Q\R\S\T\U\V\W\X\Y\Z
%   Lower-case    \a\b\c\d\e\f\g\h\i\j\k\l\m\n\o\p\q\r\s\t\u\v\w\x\y\z
%   Digits        \0\1\2\3\4\5\6\7\8\9
%   Exclamation   \!     Double quote  \"     Hash (number) \#
%   Dollar        \$     Percent       \%     Ampersand     \&
%   Acute accent  \'     Left paren    \(     Right paren   \)
%   Asterisk      \*     Plus          \+     Comma         \,
%   Minus         \-     Point         \.     Solidus       \/
%   Colon         \:     Semicolon     \;     Less than     \<
%   Equals        \=     Greater than  \>     Question mark \?
%   Commercial at \@     Left bracket  \[     Backslash     \\
%   Right bracket \]     Circumflex    \^     Underscore    \_
%   Grave accent  \`     Left brace    \{     Vertical bar  \|
%   Right brace   \}     Tilde         \~}
%



% \DoNotIndex{\newenvironment,\@bsphack,\@empty,\@esphack,\sfcode}
% \DoNotIndex{\addtocounter,\label,\let,\linewidth,\newcounter}
% \DoNotIndex{\noindent,\normalfont,\par,\parskip,\phantomsection}
% \DoNotIndex{\providecommand,\ProvidesPackage,\refstepcounter}
% \DoNotIndex{\RequirePackage,\setcounter,\setlength,\string,\strut}
% \DoNotIndex{\textbackslash,\texttt,\ttfamily,\usepackage}
% \DoNotIndex{\begin,\end,\begingroup,\endgroup,\par,\\}
% \DoNotIndex{\if,\ifx,\ifdim,\ifnum,\ifcase,\else,\or,\fi}
% \DoNotIndex{\let,\def,\xdef,\edef,\newcommand,\renewcommand}
% \DoNotIndex{\expandafter,\csname,\endcsname,\relax,\protect}
% \DoNotIndex{\Huge,\huge,\LARGE,\Large,\large,\normalsize}
% \DoNotIndex{\small,\footnotesize,\scriptsize,\tiny}
% \DoNotIndex{\normalfont,\bfseries,\slshape,\sffamily,\interlinepenalty}
% \DoNotIndex{\textbf,\textit,\textsf,\textsc}
% \DoNotIndex{\hfil,\par,\hskip,\vskip,\vspace,\quad}
% \DoNotIndex{\centering,\raggedright,\ref}
% \DoNotIndex{\c@secnumdepth,\@startsection,\@setfontsize}
% \DoNotIndex{\ ,\@plus,\@minus,\p@,\z@,\@m,\@M,\@ne,\m@ne}
% \DoNotIndex{\@@par,\DeclareOperation,\RequirePackage,\LoadClass}
% \DoNotIndex{\AtBeginDocument,\AtEndDocument}
%
% \GetFileInfo{danexam.dtx}
%
% \changes{v0.1}{2018/11/21}{Initial version}
%
%
% \title{\textsf{danexam}指南}
% \author{蛋疼的蛋蛋 \\ \texttt{sd44sd44@yeah.net}}
% \date{Version: 0.1Alpha}
%
% \maketitle \thispagestyle{empty}
% \clearpage
%
% \pagestyle{fancy}
% \tableofcontents \thispagestyle{empty}
% \clearpage
%
% \section{简介}
%
% \pkg{danexam}旨在提供一个简便易用、功能多样、囊括文理科的XeLaTeX 、LuaLaTeX 试
% 卷模版包。
%
% 本包开发过程中有参考Philip Hirschhorn所作\pkg{exam}、鲍宏昌所
% 作\pkg{BHCexam}、TJRSD所作\pkg{GEEexam},dtx文件的style参考借鉴
% 了\pkg{thuthesis},曾祥东对本包开发中遇到的问题多次提出解决方案,在此同表感谢。
%
% 作者第一次做宏包开发,多有欠缺之处还望用户多加批评指正,尤其欢迎有兴趣者与我一
% 起完善本包。
%
% \section{使用须知}
% \label{sec:xuzhi}

% \begin{enumerate}

% \item 因是中文试卷模版,请在调用本宏包前启用\pkg{ctex}。
% \begin{latex}
% \documentclass[<opt>]{ctexart}
% % 或者
% \usepackage[<opt>]{ctex}
% \end{latex}
% \note{因试卷排版的特殊性,本包将ctex的autoindent选项设为了0in,取消了段首自动
% 缩进;试卷环境不建议使用twocolumn方式;指定图片宽度尽量使用linewidth而不是
% textwidth;页面设置可由geometry命令自定义。}
%
% \item 除此之外本包构建
%   于\pkg{tasks}、\pkg{dashundergaps}、\pkg{zhlineskip}、\pkg{enumitem}等包之上。
%
% \item 事实上不可能有一个不经修改就可直接使用的试卷模版。实际使用时,可将
%   danexam.sty复制到您的源代码文件夹,并按需进行修改。
% 
% \end{enumerate}

% \section{简单示例}
% 请参照testdoc目录下各个tex源文件及与之相对应的pdf文件。
%
% \section{使用说明}
%
% \subsection{试卷标题}
% \DescribeMacro{\title}
% \DescribeMacro{\maketitle}
%
% 在试卷上打印标题(默认加粗)
% \begin{latex}
% \documentclass[zihao = 5, a4paper]{ctexart}
% \usepackage[answer]{danexam} % 显示问题答案
%
% \title{\zihao{-3} 2018年普通高等学校招生全国统一考试\\ 
%   \zihao{-2} \bfseries \heiti 理科数学}  
%
% \maketitle
%
% 你想输入的其它内容
% \end{latex}
%
% \subsection{大题标题}
% \DescribeMacro{\group}
% 设置大题的标题,默认加粗。
% \begin{latex}
% \group{大题题目内容,如“选择题。blablabla”}
% \end{latex}
%
% \subsection{题目环境}
% \DescribeEnv{qus}
% 小题枚举环境。实质上这是一个经过了配置的enumerate环境。最多可嵌套3层,第一层
% 题目标签为1. 2...;第二层为子题目,题目标签为1) 2) ...;第三层题目标签为罗马数字 i) ii) iii)....
% \begin{latex}
% \begin{qus}
%   \item 小题题目内容
%     \begin{qus}
%       \item 子题题目
%       \item 子题题目
%     \end{qus}
%   \item 小题题目内容
%   \item 小题题目内容
% \end{qus}
% \end{latex}
%
%
% \subsection{选择题}
%
% \DescribeEnv{tasks}
% \DescribeMacro{\task}
% \DescribeMacro{\choice}
%
% 本包使用\pkg{tasks}为选择题提供选项设定。\pkg{tasks}功能强大,它的具体选项及设置请在命令行运行
% \verb|texdoc tasks|查看其文档。
%
% 选项的答案请在题目内容后使用|\choice{}|命令,大括号内输入问题答案。
%
% \begin{latex}
% \group{选择题。本大题共12小题,每小题5分,共60分。blablalba}
% \begin{qus}
% \item 问题题目1 \choice{A}
%   \begin{tasks}(1)  % ()内指定选项所占用的列数,本例中为1列,也就是每一个选项占用一行。
%     \task 选项1内容 \task 选项2内容 \task 选项3内容 \task 选项4内容
%
% \item 问题题目2 \choice{C}
%   \begin{tasks}(2)  % 2列,也就是每两个选项占用一行。
%     \task 选项1内容 \task 选项2内容 \task 选项3内容 \task 选项4内容
% \end{qus}
% \end{latex}
%
% \subsection{判断题}
% \DescribeMacro{\dui}
% \DescribeMacro{\cuo}
% 为判断题提供行尾括号及答案,请在判断题目后输入。
%
% \begin{latex}
% \group{判断题。blalbla}
% \begin{qus}
%   \item 你是人类吗? \dui
%   \item 太阳从西边升起吗? \cuo
% \end{qus}
% \end{latex}
%
% \subsection{填空题}
% \DescribeMacro{\gapline}
% 填空下划线,大括号内输入答案。
%
% \begin{latex}
% \group{填空题。blablalbla}
% \begin{qus}
%   \item 人的生命只有 \gapline{一次}。 
%   \item 我们生活在 \gapline{火星}。 
% \end{qus}
% \end{latex}
%
% \DescribeMacro{\blankans}
% 不显示答案时用空白占位(无下划线)。有答案时显示答案。
%
% \begin{latex}
% \usepackage{tabularx}
% \newcolumntype{L}{>{$}X<{$}} % math-mode version of "l" column type
% …………
% \begin{qus}
% \item 直接写得数 
%   \begin{tabularx}{0.9\linewidth}{*{3}{L}}
%     3 \times 5 + 38 = \blankans{53} & 68-81 \div 9 = \blankans{59} & 52 \div 7
%     = \blankans{7 \cdots\cdots 3} \\
%   \end{tabularx}
% \end{qus}
% \end{latex}
%
% \DescribeMacro{\oneline}
% |\oneline| \oarg{length} \marg{answer}
%
% 可选参数\oarg{length}默认为填充下划线至行尾,您也可以指定下划线的长度。
%
% \begin{latex}
% \begin{qus}
%   \item 行首文字 接下划线\oneline{bla bla bla}。 
%   \item 指定下划线长度 \oneline[3cm]{blabla bla bla}。 
%   \item 独占整行需新起一行或一段 \\
%        \oneline{blabla bla bla}。 
% \end{qus}
% \end{latex}

% \subsection{简答题}
% \DescribeEnv{solution}
% 该环境有一个可选参数\oarg{length},默认为5cm。提供空间和答案。
% \begin{latex}
% \group{简答题。blablalbla}
% \begin{qus}
% \item $ 8 x 6$等于多少?
%   \begin{solution}
%     $48$
%   \end{solution}
%
% \item $ 9 x 9$等于多少?
%   \begin{solution}[4cm]
%     $81$
%   \end{solution}
% \end{qus}
% \end{latex}
%
% \subsection{作文题}
% \DescribeMacro{\drawcomposition}
% |\drawcomposition|\oarg{scale} \marg{columnnum} \marg{linenum}
% 画作文稿线。
% \begin{latex}
% \group{作文题。blablalbla}
% \begin{qus}
% \item 阅读下面文章,写作文吧。
%
%   我是一篇文章,README。
%   \drawcomposition[0.75]{20}{15} % 0.75为0.75厘米的方格,作文稿线15行20列。
% \end{qus}
% \end{latex}
%
% \note{注意命令中使用了figure和tikzpicture环境。}
%
% \section{选项说明}
% \label{sec:option}
%
% \DescribeOption{answer}
% 显示问题答案。
% \begin{latex}
% % 显示问题答案
% \usepackage[answer]{danexam}
% \end{latex}
%

% \DescribeOption{bytotal}
% 此选项默认为true,默认问题序号依次连续排序。如想让问题序号在\env{qus}结束后重新技术,请设置
% \begin{latex}
% % 每一qus 环境结束后重新排序
% \usepackage[bytotal=false]{danexam}
% \end{latex}
%
%
% \DescribeOption{lmargin}
% 显示试卷左侧密封线,本代码参考自鲍宏昌的\pkg{BHCexam}
% \begin{latex}
% % 显示密封线
% \usepackage[lmargin]{danexam}
% \end{latex}
% 
% \DescribeOption{unicodeGBMath}
% 启用unicode-math这一数学模式包,并且按照数学符号GB国标的要求进行一些调整。
% \begin{latex}
% % 使用unicode-math,并对个别数学符号微调。
% \usepackage[unicodeGBMath]{danexam}
% \end{latex}
% \note{试验选项,主要为作者个人自用。开启本选项则强行启用\pkg{unicode-math}包,
% 并为兼容国标对数学符号外观做些调整,详情请见\pkg{unicode-math}文档及本代码所借鉴
% 文章—— Zeping Lee所写的
% \href{https://github.com/latexstudio/ChinaThesis/wiki/GB-math}{《LaTeX 排版国标样
% 式的数学符号》}}
%
% % \StopEventually{}
%
% \section{源代码说明}
% 源代码过于简单,不许再做过多说明,请直接参看\file{danexam.sty}文件。

%    \begin{macrocode}
%<*sty>
\NeedsTeXFormat{LaTeX2e}
\ProvidesPackage{danexam}[%
  2018/10/28 v1.0c a simple exam package]

  \ctexset{autoindent = 0pt}
\RequirePackage{kvoptions}

\SetupKeyvalOptions{
  family=dan,
  prefix=dan@
}

\DeclareBoolOption[false]{answer}
\DeclareBoolOption[true]{bytotal}
\DeclareBoolOption[false]{lmargin}
\DeclareBoolOption[false]{unicodeGBMath}

\DeclareDefaultOption{%
  \PackageError{\@currname}{%
    Unknown option `\CurrentOption'%
  }{%
    Available package options are:\MessageBreak
    \space\space`answer,\MessageBreak
    \space\space`bytotal,\MessageBreak
    \space\space`lmargin,\MessageBreak
    \space\space`unicodeGBMath%
  }%
}

\ProcessKeyvalOptions*

\RequirePackage{amsmath,dashundergaps}

  

\ifdan@unicodeGBMath
  % 注意,unicode-math与被其认为过时的bm包不兼容,不要\RequirePackage{bm}
  \RequirePackage[math-style=ISO, bold-style=ISO, partial = upright]{unicode-math}%
  \setmathfont[Scale = 1.1]{Libertinus Math}  
% \AtBeginDocument{% add \mathbb  方法二
% \let\mathbb\relax
% \DeclareMathAlphabet{\mathbb}{U}{msb}{m}{n}%
% }
%  \setmathfont[range=\mathbb]{texgyrepagella-math.otf} % 方法三

% \setsansfont{TeX Gyre Heros}
% \setmonofont[Scale=1]{Libertinus Mono}
  % 以下修改根据zeping Lee 的文章而来
  % https://github.com/latexstudio/ChinaThesis/wiki/GB-math
  \AtBeginDocument{\renewcommand\parallel{\mathrel{/\mskip-2.5mu/}}}
  \global\let\sim\backsim
  \global\let\cong\backcong
  \renewcommand\mathellipsis{\cdots}
  \global\let\Delta\increment
  % \let\limU\removenolimits  %limit position up
  \AtBeginDocument{\renewcommand{\Re}{\operatorname{Re}}}
  \AtBeginDocument{\renewcommand{\Im}{\operatorname{Im}}}
  \AtBeginDocument{\renewcommand{\leq}{\leqslant}}
  \AtBeginDocument{\renewcommand{\geq}{\geqslant}}
\fi

\RequirePackage{scalerel}

% \setmainfont[Scale=1.1, SlantedFont = {Libertinus Serif Italic}]{Libertinus Serif}

\newsavebox{\mlbox}
\sbox{\mlbox}{
	\begin{minipage}{0.9\paperheight}
		\begin{center}
			\heiti
			班级 \underline{\hspace{15ex}} \quad
			姓名 \underline{\hspace{15ex}} \quad
			学号 \underline{\hspace{15ex}} \quad \\
			\vspace{3ex}
			\dotfill \dotfill 装 \dotfill 订 \dotfill 线 \dotfill \dotfill
		\end{center}
	\end{minipage}
}

\newcommand{\printml}{
  \reversemarginpar
  \addtolength{\oddsidemargin}{0.3cm}
	\marginpar{\rotatebox{90}{\usebox{\mlbox}}}
}


\RequirePackage[bodytextleadingratio = 2, restoremathleading = true, footnoteleadingratio = 1.67]{zhlineskip}

% TODO: 留待日后或许使用。
% https://tex.stackexchange.com/questions/279652/macro-like-ignorespaces-for-ignoring-pars
% \long\def\eat#1{\ignorepars}
% \def\ignorepars{\futurelet\next\ignoreparsA}
% \def\ignoreparsA{\ifx\next\par\expandafter\eat\fi}

% 不好的习惯,但却是习俗。
\everymath{\displaystyle}

\dashundergapssetup{
  gap-numbers = false,
  gap-widen = true,
  gap-extend-minimum = 10pt,
  gap-extend-percent = 20,
  teacher-gap-format = underline,
  gap-font = \bfseries \large,
}

\newcommand{\blankans}[1]{%
  \large \relax\ifmmode \else \bfseries \fi 
  \ifdan@answer
    (#1) 
  \else
    \phantom{#1}
  \fi
}

% 如果gapline参数为空的话,则显示两个 \qquad
\def \if@empty#1{\def\temp{#1} \ifx\temp\empty }
\newcommand{\gapline}[1]{
  \ifdan@answer
  \dashundergapssetup{teacher-mode = true}
  \else
  \dashundergapssetup{teacher-mode = false}
  \fi%
  \hskip 1pt%
  \if@empty{#1} \gap{\qquad \qquad}
  \else
  \relax\ifmmode \dashundergapssetup{gap-font= \null} \fi \gap{#1} 
  \fi \dashundergapssetup{gap-font= \bfseries \large} \hskip 1pt%
}

\usepackage{linegoal}
\newcommand{\oneline}[2][\linegoal]{%
  \CJKunderline[hidden = false, skip = false]{\makebox[#1][c]{\large \bfseries
      \ifdan@answer #2 \fi}}
}

\RequirePackage{geometry}

% TODO:以后做其他页面支持时save restore各种geometry。


\geometry{%
  includemp = false, % includes the margin notes, \marginparwidth and \marginparsep, into body when calculating horizontal calculation.
  marginparsep = 0.3cm,
  marginparwidth = 2cm,
  % bindingoffset = 1cm,
  top = 2cm,
  bottom = 3cm,
  outer = 2.5cm, 
  inner = 3cm,
  headheight = 6mm,
  headsep = 5mm,
  footskip = 10mm,
}

\RequirePackage{fancyhdr, lastpage}
\pagestyle{fancy}
\fancyhf{} % clear all header and footer fields
\cfoot{\small \kaishu \quad 第\,\thepage\,页 \quad (共\pageref{LastPage}页) } %照抄他人
% 代码

% 以下twocolumn设置参考夜神月的设置
% https://my.oschina.net/u/3468093/blog/2249279
\if@twocolumn
  \geometry{columnsep = 1.3cm}
  \def\lastfox@putlabel{%
    \immediate\write\@auxout{%
      \string\newlabel{LastFox}{{\@arabic\c@fox}{}{}{}{}}}}
  \AtEndDocument{\lastfox@putlabel}
  \newcounter{foo}
  \newcounter{fox}
  \addtocounter{foo}{1}
  \fancyfoot[CE,CO]{\kaishu{}第\refstepcounter{fox}\thefoo\refstepcounter{foo}页~/~共~\ref{LastFox}~页\hspace*{18cm}第\refstepcounter{fox}\thefoo\refstepcounter{foo}页~/~共~\ref{LastFox}~页}
\fi



\renewcommand{\headrulewidth}{0pt}

\RequirePackage{enumitem}
% \topsep 列表顶部与之前内容的额外空白,不含 \baselineskip
% \partopsep 如果列表之前是一个空行,列表顶部的额外空白
% \itemsep  列表各项之间额外的垂直空白
% \parsep 一个 item 中,如果分段,段落间额外空白
% \leftmargin 列表与左边距之间的水平距离,值为非负
% \rightmargin 列表与右边距之间的水平距离,值为非负
% \itemindent 每一 item 第一行的缩进
% \listparindent 每一 item 第一行之后各行的缩进
% \labelsep 标签盒子与每一 item 第一行文本之间距离
% \labelwidth 标签盒子的宽度;如果标签过长,这一宽度会自动变大,直到列表的第一行文本为止

% TODO:过时代码,留存@Question和@Ingroupqs这两个counter,留待以后使用。

\setlist{itemindent = \parindent, listparindent =
  \parindent, %labelindent = \parindent,
  parsep = 0ex, partopsep = 0ex, itemsep = 0em }

\newcounter{@Group}
\newcounter{@Question}
\newcounter{@Ingroupqs}[@Group]

% TODO:过时代码,留存@Question和@Ingroupqs这两个counter,留待以后使用。
\newcommand{\addquscounter}{
  \stepcounter{@Question}
  \stepcounter{@Ingroupqs}
  \ifdan@bytotal
  \renewcommand{\the@Question}{\arabic{@Question}}
  \else 
  \renewcommand{\the@Question}{\arabic{@Ingroupqs}}
  \fi%
}

% TODO:期待enumerate早日完成 \AddEnumerateCounter的新实现。
\newlist{qus}{enumerate}{3}
\setlist[qus]{leftmargin = *, partopsep = 0em, topsep = 0em, 
  parsep = 5pt, itemsep = 0em, leftmargin = 1.4em, listparindent = 0em}

% setlist在目前版本不能叠加效果。
\ifdan@bytotal
  \setlist[qus, 1]{label = {\arabic*}., resume}
\else
  \setlist[qus, 1]{label = {\arabic*}.}
\fi

\setlist[qus, 2]{label = {\arabic*})}
\setlist[qus, 3]{label = {\roman*})}

\newcommand{\group}[1]{
  % 仿section的空间设定。
  % \addvspace{3.5ex plus 1ex minus .2ex}
  \addvspace{3.5ex plus 2ex minus .2ex}
  \stepcounter{@Group}
  {\bfseries \chinese{@Group}、#1 \par}
  \vskip 0.5em%
}


\renewcommand*{\title}[1]{\gdef\@title{#1}}

\renewcommand\maketitle{
  \newpage \null
  \ifdan@lmargin \printml \fi
  \vspace{-1cm}
    {\centering \bfseries\@title \par}%
  \par
  \noindent \unskip
}


\newcommand{\fillline}[1]{%
  \hspace*{\fill}\mbox{(\quad#1\quad)}%
}
\RequirePackage{pifont}
\newcommand{\dui}{
  \ifdan@answer \fillline{\large \ding{52}}
  \else \fillline{\quad}
  \fi
}

\newcommand{\cuo}{
  \ifdan@answer \fillline{\large \ding{56}}
  \else \fillline{\quad}
  \fi
}

\newcommand{\choice}[1]{%
  \ifdan@answer \fillline{\large \bfseries #1}
  \else \fillline{\quad}
  \fi
}

% ------------------------------ 选择题 ----------------------------------%
% 本代码来自于天津大学的TJRSD
% https://github.com/TJRSD/gaokao_exam/blob/master/GEEexam.sty
\RequirePackage{tasks}%选择题宏包,tasks环境
\settasks{
  counter-format={tsk[A].},
  label-offset={0.4em},
  label-align=left,
  label-width = 1.3em,
  column-sep={2pt},
  item-indent={1pt},
  before-skip={-0.3em},
  after-skip={0em},
  % after-item-skip = 1ex plus 1ex minus 1ex %  default skip
  after-item-skip = 0ex plus 1pt minus 1pt
}


% ------------------------------ 简答题 ----------------------------------%


% TODO: solution,不完善,先占位。 

\usepackage{environ}
\NewEnviron{solution}[1][5cm]{%
  \par
  \begin{minipage}[t][#1][t]{\linewidth}
  \ifdan@answer
    \large \textbf{答案}:\vskip 1ex
  \fi
  \end{minipage}
}

\RequirePackage{tikz}
\usetikzlibrary{shapes,snakes}

% 代码来源于http://www.latexstudio.net/archives/6609.html
% TODO:线条的占用了内部空间,使内部空间不是完全尺寸。
\newcommand{\drawcomposition}[3][0.8]{
  % 使用minipage,搭配linewidth选项可防止twolumn时出现跨栏错误,但在multicols等
  % 环境下会出问题。最好还是不要用任何twocolumn。
  \begin{figure}[!ht]
    \centering
    \begin{tikzpicture}[scale=#1]
      % 画#1条竖线代表单行#1个格
      \foreach \m in {1,2,...,#2}\draw(\m,0) -- (\m,{#3 *1.4 - 0.4});
      % 画#2个横矩形,代表#2行
      \foreach \n in {1,2,..., \the\numexpr#3 }\draw[fill=white,line width=0.4pt](0,{1.4*\n-0.4})rectangle(#2 ,{1.4*\n});
      % 以下#1*#2个方格,如果能定义常量就方便操作了。
      \draw[line width = 0.8pt](0,0)rectangle(#2,{#3 *1.4});
    \end{tikzpicture}
  \end{figure}
}

%</sty>
%    \end{macrocode}


% \iffalse
%    \begin{macrocode}
%<*dandanstyle>

\ProvidesPackage{dandanstyle}
\RequirePackage{hypdoc}
\RequirePackage[UTF8,scheme=chinese,fontset=sourcesans]{ctex}

\setmainfont[Scale=1.1, SlantedFont = {Libertinus Serif Italic}]{Libertinus Serif}
\setsansfont{TeX Gyre Heros}
\setmonofont[Scale=1]{Libertinus Mono}

\RequirePackage[math-style=ISO, bold-style=ISO, partial = upright]{unicode-math}%
\setmathfont[Scale = 1.1]{Libertinus Math}  
% \setsansfont{TeX Gyre Heros}
% \setmonofont[Scale=1]{Libertinus Mono}
% 以下修改根据zeping Lee 的文章而来
% https://github.com/latexstudio/ChinaThesis/wiki/GB-math
\AtBeginDocument{\renewcommand\parallel{\mathrel{/\mskip-2.5mu/}}}
\global\let\sim\backsim
\global\let\cong\backcong
\renewcommand\mathellipsis{\cdots}
\global\let\Delta\increment
% \let\limU\removenolimits  %limit position up
\AtBeginDocument{\renewcommand{\Re}{\operatorname{Re}}}
\AtBeginDocument{\renewcommand{\Im}{\operatorname{Im}}}
\AtBeginDocument{\renewcommand{\leq}{\leqslant}}
\AtBeginDocument{\renewcommand{\geq}{\geqslant}}
\RequirePackage{geometry}
\geometry{a4paper, inner= 4cm, outer = 2.8cm}
\RequirePackage{array,longtable,booktabs}
\RequirePackage{listings}
\RequirePackage{fancyhdr}
\RequirePackage[table]{xcolor}
\RequirePackage{enumitem}
\RequirePackage{etoolbox}
\RequirePackage{metalogo}

\colorlet{thu@macro}{blue!60!black}
\colorlet{thu@env}{blue!70!black}
\colorlet{thu@option}{purple}
\patchcmd{\PrintMacroName}{\MacroFont}{\MacroFont\bfseries\color{thu@macro}}{}{}
\patchcmd{\PrintDescribeMacro}{\MacroFont}{\MacroFont\bfseries\color{thu@macro}}{}{}
\patchcmd{\PrintDescribeEnv}{\MacroFont}{\MacroFont\bfseries\color{thu@env}}{}{}
\patchcmd{\PrintEnvName}{\MacroFont}{\MacroFont\bfseries\color{thu@env}}{}{}

\def\DescribeOption{%
  \leavevmode\@bsphack\begingroup\MakePrivateLetters%
  \Describe@Option}
\def\Describe@Option#1{\endgroup
  \marginpar{\raggedleft\PrintDescribeOption{#1}}%
  \thu@special@index{option}{#1}\@esphack\ignorespaces}
\def\PrintDescribeOption#1{\strut \MacroFont\bfseries\sffamily\color{thu@option} #1\ }
\def\thu@special@index#1#2{\@bsphack
  \begingroup
    \HD@target
    \let\HDorg@encapchar\encapchar
    \edef\encapchar usage{%
      \HDorg@encapchar hdclindex{\the\c@HD@hypercount}{usage}%
    }%
    \index{#2\actualchar{\string\ttfamily\space#2}
           (#1)\encapchar usage}%
    \index{#1:\levelchar#2\actualchar
           {\string\ttfamily\space#2}\encapchar usage}%
  \endgroup
  \@esphack}

\lstdefinestyle{lstStyleBase}{%
   basicstyle=\small\sffamily,
   aboveskip=\medskipamount,
   belowskip=\medskipamount,
   lineskip=0pt,
   boxpos=c,
   showlines=false,
   extendedchars=true,
   upquote=true,
   tabsize=2,
   showtabs=false,
   showspaces=false,
   showstringspaces=false,
   numbers=none,
   linewidth=\linewidth,
   xleftmargin=4pt,
   xrightmargin=0pt,
   resetmargins=false,
   breaklines=true,
   breakatwhitespace=false,
   breakindent=0pt,
   breakautoindent=true,
   columns=flexible,
   keepspaces=true,
   gobble=2,
   framesep=3pt,
   rulesep=1pt,
   framerule=1pt,
   backgroundcolor=\color{gray!5},
   stringstyle=\color{green!40!black!100},
   keywordstyle=\bfseries\color{blue!50!black},
   commentstyle=\slshape\color{black!60}}

\lstdefinestyle{lstStyleShell}{%
   style=lstStyleBase,
   frame=l,
   rulecolor=\color{purple},
   language=bash}

\lstdefinestyle{lstStyleLaTeX}{%
   style=lstStyleBase,
   frame=l,
   rulecolor=\color{violet},
   language=[LaTeX]TeX}

\lstnewenvironment{latex}{\lstset{style=lstStyleLaTeX}}{}
\lstnewenvironment{shell}{\lstset{style=lstStyleShell}}{}

\setlist{nosep}

\DeclareDocumentCommand{\option}{m}{\textsf{#1}}
\DeclareDocumentCommand{\env}{m}{\texttt{#1}}
\DeclareDocumentCommand{\pkg}{s m}{%
  \texttt{#2}\IfBooleanF#1{\thu@special@index{package}{#2}}}
\DeclareDocumentCommand{\file}{s m}{%
  \texttt{#2}\IfBooleanF#1{\thu@special@index{file}{#2}}}
\newcommand{\myentry}[1]{%
  \marginpar{\raggedleft\color{purple}\bfseries\strut #1}}
\newcommand{\note}[2][注意:]{{%
  \color{magenta}{\bfseries #1}\emph{#2}}}

\def\thuthesis{\textsc{Thu}\-\textsc{Thesis}}

\RequirePackage{xargs}
\RequirePackage[colorinlistoftodos,prependcaption,textsize=small]{todonotes}
\newcommandx{\unsure}[2][1=]{\todo[linecolor=red,backgroundcolor=red!25,bordercolor=red,#1]{#2}}
\newcommandx{\change}[2][1=]{\todo[linecolor=blue,backgroundcolor=blue!25,bordercolor=blue,#1]{#2}}
\newcommandx{\info}[2][1=]{\todo[linecolor=OliveGreen,backgroundcolor=OliveGreen!25,bordercolor=OliveGreen,#1]{#2}}
\newcommandx{\improve}[2][1=]{\todo[linecolor=Plum,backgroundcolor=Plum!25,bordercolor=Plum,#1]{#2}}
\newcommandx{\thiswillnotshow}[2][1=]{\todo[disable,#1]{#2}}


%%
%</dandanstyle>
%    \end{macrocode}
% \fi

% \Finale
%
% \setcounter{IndexColumns}{2}
%
% \def\indexname{索引}
% \def\glossaryname{修改记录}
% \IndexPrologue{\section{索引}}
% \GlossaryPrologue{\section{版本更新}}
%
% \PrintIndex \PrintChanges
\endinput

